\section{Исследование управляемости}

Рассмотрим систему $\dot{x} = Ax + Bu$, где 
\begin{equation}
    \begin{array}{cc}
        A = \begin{bmatrix}
            5 & -2 & 8 \\
            4 & -3 & 4 \\
            -4 & 0 & -7
        \end{bmatrix}, &
        B = \begin{bmatrix}
            -7 \\
            -5 \\
            7
        \end{bmatrix}
    \end{array}
\end{equation}

\subsection{Управляемость системы}
\subsubsection{Матрица управляемости}
Найдем матрицу управляемости $U = [B, AB, A^2B]$: 
\begin{equation}
    U = \begin{bmatrix} 
        \begin{array}{c|c|c}
            \begin{bmatrix}
                -7 \\
                -5 \\
                7
            \end{bmatrix} & 
            \begin{bmatrix}
                5 & -2 & 8 \\
                4 & -3 & 4 \\
                -4 & 0 & -7
            \end{bmatrix} \times 
            \begin{bmatrix}
                -7 \\
                -5 \\
                7
            \end{bmatrix} &
            \begin{bmatrix}
                5 & -2 & 8 \\
                4 & -3 & 4 \\
                -4 & 0 & -7
            \end{bmatrix}^2 \times
            \begin{bmatrix}
                -7 \\
                -5 \\
                7
            \end{bmatrix}
        \end{array}   
    \end{bmatrix}
\end{equation}
\begin{equation}
    U = \begin{bmatrix}
        -7 & 31 & -43 \\
        -5 & 15 & -5 \\
        7 & -21 & 23 \\
    \end{bmatrix}
\end{equation}
Определим ранг матрицы управляемости:
\begin{equation}
    \text{rank}(U) = 3
\end{equation}
Так как ранг матрицы управляемости равен порядку системы, то система является полностью управляемой согласно критерию Калмана.

\subsubsection{Управляемость собственных значений}
Найдем спектр матрицы $A$:
\begin{equation}
    \sigma(A) = \{-3, -1-2j, -1+2j\}
\end{equation}

Для каждого собственного значения найдем матрицу Хаутуса $H_i = \begin{bmatrix} A - \lambda_i I & B \end{bmatrix}$ и определим ее ранг:
\begin{enumerate}
    \item $\lambda_1 = -3$: $H_1 = \begin{bmatrix}
        8 & -2 & 8 & -7\\
        4 & 0 & 4 & -5 \\
        -4 & 0 & -4 & 7
    \end{bmatrix}$, $\text{rank}(H_1) = 3$, собственное значение управляемо.
    \item $\lambda_2 = -1-2j$: $H_2 = \begin{bmatrix}
        6+2j & -2 & 8 & -7\\
        4 & -2+2j & 4 & -5 \\
        -4 & 0 & -6+2j & 7
    \end{bmatrix}$, $\text{rank}(H_2) = 3$, собственное значение управляемо.
    \item $\lambda_3 = -1+2j$: $H_3 = \begin{bmatrix}
        6-2j & -2 & 8 & -7\\
        4 & -2-2j & 4 & -5 \\
        -4 & 0 & -6-2j & 7
    \end{bmatrix}$, $\text{rank}(H_3) = 3$, собственное значение управляемо.
\end{enumerate}
Так как выше было показано, что система является полностью управляемой, то каждое собственное значение матрицы $A$ является управляемым. 

\subsubsection{Диагональная форма системы}
Найдем диагональную форму системы, заменив базис на базис из собственных векторов матрицы $A$:
\begin{equation}
    \dot{\hat{x}} = P^{-1}AP\hat{x} + P^{-1}Bu
\end{equation}
Где $P$ -- матрица собственных векторов матрицы $A$. 
Найдем собственные векторы матрицы $A$:
\begin{equation}
    \begin{array}{ccc}
        v_1 = \begin{bmatrix} -1 \\ 0 \\ 1 \end{bmatrix} &
        v_2 = \begin{bmatrix} -3+j \\ -2 \\ 2 \end{bmatrix} &
        v_3 = \begin{bmatrix} -3-j \\ -2 \\ 2 \end{bmatrix} 
    \end{array}
\end{equation}
Тогда матрица $P$:
\begin{equation}
    P = \begin{bmatrix}
        -1 & -3+j & -3-j \\
        0 & -2 & -2 \\
        1 & 2 & 2
    \end{bmatrix}
\end{equation}
Система преобразуется к виду:
\begin{equation}
    \dot{\hat{x}} = \begin{bmatrix}
        -3 & 0 & 0 \\
        0 & -1-2j & 0 \\
        0 & 0 & -1+2j
    \end{bmatrix} \hat{x} + 
    \begin{bmatrix}
        2 \\
        \frac{5 - 5j}{4} \\ 
        \frac{5 + 5j}{4}
    \end{bmatrix} u
\end{equation}
Так как все элементы $P^{-1}B$ не равны нулю, то система является полностью управляемой, каждая мода системы управляема.

\subsection{Грамиан управляемости}
Найдем грамиан управляемости $P(t_1)$:
\begin{equation}
    P(t_1) = \int_{0}^{t_1} e^{At}BB^Te^{A^Tt}dt
\end{equation}
Вычислим грамиан управляемости для $t_1 = 3$ с помощью функции \texttt{gram}: 
\begin{equation}
    P(3) = \begin{bmatrix}
        18.12 & 10.97 & -11.64 \\ 
        10.97 & 7.48 & -8.48 \\ 
        -11.64 & -8.48 & 10.14 \\ 
    \end{bmatrix}
\end{equation}
Найдем собственные числа Грамиана управляемости: 
\begin{equation}
    \sigma(P(3)) = \{ 0.05, 1.94, 33.74 \}
\end{equation}
Все собственные числа Грамиана управляемости положительны, что говорит о том, что система является управляемой.


\subsection{Управление системой}
Найдем управление $u(t)$, которое будет переводить систему из состояния $x(0) = 0$ в состояние $x_1 = x(t_1) = \begin{bmatrix} -2 & -3 & 3 \end{bmatrix}^T$. 
\begin{equation}
    u(t) = B^Te^{A^T(t_1 - t)}P(t_1)^{-1}x_1
\end{equation}
Реализуем данное управление в MATLAB и проведем моделирование системы. 
\begin{figure}
    \centering
    \includegraphics[width=\textwidth]{media/plots/task1_control_signal.png}
    \caption{Управление системой}
    \label{fig:task1_control_signal}
\end{figure}
На рисунке \ref{fig:task1_control_signal} изображено управление системой.
\begin{figure}
    \centering
    \includegraphics[width=\textwidth]{media/plots/task1_states.png}
    \caption{Состояние системы}
    \label{fig:task1_state}
\end{figure}
На рисунке \ref{fig:task1_state} изображено состояние системы.

Видно, что система управляемая в соответствии с заданным управлением и переходит в заданное состояние. 
\FloatBarrier
\subsection{Вывод}
При исследовании системы, рассматриваемой в этом заднии, удалось показать, что она является 
полностью управляемой. Это было продемонстрировано с помощью критерия Калмана, через 
управляемость собственных значений и диагональную форму системы. Также был найден грамиан 
управляемости и проверены его собственные числа. Проведено моделирование системы с управлением, 
которое переводит систему в заданное состояние. Результаты моделирования показали, что система
управляема и управление работает корректно. 
\FloatBarrier
\section{Реальное дифференцирующее звено}
В регуляторе (\ref{eq:conctoller1}) заменим идеальное дифференцирующее звено на реальное дифференцирующее звено -- 
передаточную функцию вида: 
\begin{equation}
    W(s) = \frac{s}{Ts + 1}
\end{equation}
И найдем его передаточную функцию:
\begin{equation}
    W_{y\rightarrow u}(s) = \frac{k_0(Ts + 1) + k_1s}{Ts + 1} = \frac{(k_0T + k_1)s + k_0}{Ts + 1}
\end{equation}
Запишем передаточную функцию разомкнутой системы:
\begin{equation}
    W_{s}(s) = \frac{1}{a_2s^2 + a_1s + a_0} 
\end{equation}
Теперь найдем передаточную функцию замкнутой системы:
\begin{multline}
    W_{u\rightarrow y}(s) = \frac{W_s(s)}{1 - W_s(s)W_{y\rightarrow u}(s)} = \frac{\frac{1}{a_2s^2 + a_1s + a_0} }{1 - \frac{1}{a_2s^2 + a_1s + a_0} \cdot \frac{(k_0T + k_1)s + k_0}{Ts + 1}} = \\
    \frac{1}{a_2s^2 + a_1s + a_0 - \frac{(k_0T + k_1)s + k_0}{Ts + 1}} = \frac{Ts + 1}{(a_2s^2 + a_1s + a_0)(Ts + 1) - (k_0T + k_1)s - k_0} = \\
    \frac{Ts + 1}{a_2Ts^3 + (a_2 + a_1T)s^2 + (a_1 + a_0T - k_0T - k_1)s + (a_0 - k_0)}
\end{multline}
Снова воспользуемся критерием Гурвица для определения границы устойчивости системы.
\begin{equation}
    \begin{cases}
        \frac{a_2 + a_1T}{a_2T} > 0 \\
        \frac{a_1 + a_0T - k_0T - k_1}{a_2T} > 0 \\
        \frac{a_0 - k_0}{a_2T} > 0 \\
        \frac{a_2 + a_1T}{a_2T}\cdot\frac{a_1 + a_0T - k_0T - k_1}{a_2T} > \frac{a_0 - k_0}{a_2T}
    \end{cases} \Rightarrow 
    \begin{cases}
        \frac{1 -T}{T} > 0 \\
        \frac{-1 -2T +3T +3}{T} > 0 \\
        \frac{-2 +3}{T} > 0 \\
        \frac{1 -T}{T}\cdot\frac{-1 -2T +3T +3}{T} > \frac{-2 +3}{T}
    \end{cases} \Rightarrow
    \begin{cases}
        T \in (0, 1) \\
        T \in (0, \infty) \\
        T \in (0, \infty) \\
        T \in (0, \sqrt{3} - 1)
    \end{cases}
\end{equation}
\begin{equation}
    T \in (0, \sqrt{3} - 1)
\end{equation}

\subsection{Моделирование системы}
Промоделируем систему, заменив производную на передаточную функцию (см. рис. \ref{fig:task2_scheme}).
\begin{figure}[ht!]
    \centering
    \includegraphics[width=\textwidth]{"media/scheme2.png"}
    \caption{Схема моделирования системы}
    \label{fig:task2_scheme}
\end{figure}
% Для начала возьмем $T = 0.73 \approx \sqrt{3} - 1$. Таким образом, согласно теоретическим расчетам, система будет устойчива.
% должна быть близка к границе устойчивости. Промоделируем (см рис. \ref{fig:task2_out}).
% \begin{figure}[ht!]
%     \centering
%     \includegraphics[width=\textwidth]{"media/plots/task2_out.png"}
%     \caption{Свободное движение системы ($T = 0.73$)}
%     \label{fig:task2_out}
% \end{figure}
% Теоретические ожидания подтвердились. Система близка к границе устойчивости, но все еще является устойчивой, 
% так как было выбрана значение несколько меньшее, чем $\sqrt{3} - 1$. Теперь выберем значение $T = 0.74$ и промоделируем систему (см. рис. \ref{fig:task2_out2}).
% \begin{figure}[ht!]
%     \centering
%     \includegraphics[width=\textwidth]{"media/plots/task2_out2.png"}
%     \caption{Свободное движение системы ($T = 0.74$)}
%     \label{fig:task2_out2}
% \end{figure}
% Теперь видно, что система стала неустойчивой, что подтверждает теоретические расчеты.

% Выберем меньшее значение $T = 0.001$ и промоделируем систему (см. рис. \ref{fig:task2_out3}).
% \begin{figure}[ht!]
%     \centering
%     \includegraphics[width=\textwidth]{"media/plots/task2_out3.png"}
%     \caption{Свободное движение системы ($T = 0.001$)}
%     \label{fig:task2_out3}
% \end{figure}

Выберем значения $T = \{0.73, 0.74, 0.001\}$. В первом случае система будет устойчива и близка к границе устойчивости,
во втором случае система будет неустойчива, а в третьем случае система будет устойчива, но менее колебательна.
Результаты моделирования приведены на рис. \ref{fig:task2_out}, \ref{fig:task2_err}.
\begin{figure}[ht!]
    \centering
    \includegraphics[width=\textwidth]{media/plots/task2_out.png}
    \caption{Графики выходного сигнала}
    \label{fig:task2_out}
\end{figure}

\begin{figure}
    \centering
    \includegraphics[width=\textwidth]{media/plots/task2_error.png}
    \caption{Графики ошибки}
    \label{fig:task2_err}
\end{figure}


\subsection{Вывод}
В данном разделе было проведено моделирование системы с реальным дифференцирующим звеном. 
Система оказалась устойчивой при $T \in (0, \sqrt{3} - 1)$, что подтверждает теоретические расчеты. 
При уменьшении значения $T$ система становится менее колебательной, так как передаточна 
функция дифференцирующего звена приближается к идеальному дифференцирующему звену. 
\FloatBarrier

\section{Выводы}
В лабораторной работе было рассмотрено вынужденное движение системы и характеристики переходных процессов.
Были рассмотрены корневые критерии качества и проведено сравнение систем с разными коэффициентами.

Моделирование систем с различными входные сигналами показало, что 
даже устойчивая система может не сходится к установившемуся значению при некоторых входных сигналах. 
Начальные условия так же влияют на поведение системы, но глобальная структура переходного процесса остается неизменной.

Результаты симуляции показали, что при увеличении максимальной величины действительной части корней 
характеристического уравнения, время переходного процесса уменьшается.
При этом, перерегулирование зависит от колебательности системы, которая пропорциональна значению $\mu = \text{max}\left|\frac{Im(\lambda)}{Re(\lambda)}\right|$.

