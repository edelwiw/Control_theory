\section{Исследование управляемости}

Рассмотрим систему $\dot{x} = Ax + Bu$, где 
\begin{equation}
    \begin{array}{cc}
        A = \begin{bmatrix}
            5 & -2 & 8 \\
            4 & -3 & 4 \\
            -4 & 0 & -7
        \end{bmatrix}, &
        B = \begin{bmatrix}
            -7 \\
            -5 \\
            7
        \end{bmatrix}
    \end{array}
\end{equation}

\subsection{Управляемость системы}
\subsubsection{Матрица управляемости}
Найдем матрицу управляемости $U = [B, AB, A^2B]$: 
\begin{equation}
    U = \begin{bmatrix} 
        \begin{array}{c|c|c}
            \begin{bmatrix}
                -7 \\
                -5 \\
                7
            \end{bmatrix} & 
            \begin{bmatrix}
                5 & -2 & 8 \\
                4 & -3 & 4 \\
                -4 & 0 & -7
            \end{bmatrix} \times 
            \begin{bmatrix}
                -7 \\
                -5 \\
                7
            \end{bmatrix} &
            \begin{bmatrix}
                5 & -2 & 8 \\
                4 & -3 & 4 \\
                -4 & 0 & -7
            \end{bmatrix}^2 \times
            \begin{bmatrix}
                -7 \\
                -5 \\
                7
            \end{bmatrix}
        \end{array}   
    \end{bmatrix}
\end{equation}
\begin{equation}
    U = \begin{bmatrix}
        -7 & 31 & -43 \\
        -5 & 15 & -5 \\
        7 & -21 & 23 \\
    \end{bmatrix}
\end{equation}
Определим ранг матрицы управляемости:
\begin{equation}
    \text{rank}(U) = 3
\end{equation}
Так как ранг матрицы управляемости равен порядку системы, то система является полностью управляемой согласно критерию Калмана.

\subsubsection{Управляемость собственных значений}
Найдем спектр матрицы $A$:
\begin{equation}
    \sigma(A) = \{-3, -1-2j, -1+2j\}
\end{equation}

Для каждого собственного значения найдем матрицу Хаутуса $H_i = \begin{bmatrix} A - \lambda_i I & B \end{bmatrix}$ и определим ее ранг:
\begin{enumerate}
    \item $\lambda_1 = -3$: $H_1 = \begin{bmatrix}
        8 & -2 & 8 & -7\\
        4 & 0 & 4 & -5 \\
        -4 & 0 & -4 & 7
    \end{bmatrix}$, $\text{rank}(H_1) = 3$, собственное значение управляемо.
    \item $\lambda_2 = -1-2j$: $H_2 = \begin{bmatrix}
        6+2j & -2 & 8 & -7\\
        4 & -2+2j & 4 & -5 \\
        -4 & 0 & -6+2j & 7
    \end{bmatrix}$, $\text{rank}(H_2) = 3$, собственное значение управляемо.
    \item $\lambda_3 = -1+2j$: $H_3 = \begin{bmatrix}
        6-2j & -2 & 8 & -7\\
        4 & -2-2j & 4 & -5 \\
        -4 & 0 & -6-2j & 7
    \end{bmatrix}$, $\text{rank}(H_3) = 3$, собственное значение управляемо.
\end{enumerate}
Так как выше было показано, что система является полностью управляемой, то каждое собственное значение матрицы $A$ является управляемым. 

\subsubsection{Диагональная форма системы}
Найдем диагональную форму системы, заменив базис на базис из собственных векторов матрицы $A$:
\begin{equation}
    \dot{\hat{x}} = P^{-1}AP\hat{x} + P^{-1}Bu
\end{equation}
Где $P$ -- матрица собственных векторов матрицы $A$. 
Найдем собственные векторы матрицы $A$:
\begin{equation}
    \begin{array}{ccc}
        v_1 = \begin{bmatrix} -1 \\ 0 \\ 1 \end{bmatrix} &
        v_2 = \begin{bmatrix} -3+j \\ -2 \\ 2 \end{bmatrix} &
        v_3 = \begin{bmatrix} -3-j \\ -2 \\ 2 \end{bmatrix} 
    \end{array}
\end{equation}
Тогда матрица $P$:
\begin{equation}
    P = \begin{bmatrix}
        -1 & -3+j & -3-j \\
        0 & -2 & -2 \\
        1 & 2 & 2
    \end{bmatrix}
\end{equation}
Система преобразуется к виду:
\begin{equation}
    \dot{\hat{x}} = \begin{bmatrix}
        -3 & 0 & 0 \\
        0 & -1-2j & 0 \\
        0 & 0 & -1+2j
    \end{bmatrix} \hat{x} + 
    \begin{bmatrix}
        2 \\
        \frac{5 - 5j}{4} \\ 
        \frac{5 + 5j}{4}
    \end{bmatrix} u
\end{equation}
Так как все элементы $P^{-1}B$ не равны нулю, то система является полностью управляемой, каждая мода системы управляема.

\subsection{Грамиан управляемости}
Найдем грамиан управляемости $P(t_1)$:
\begin{equation}
    P(t_1) = \int_{0}^{t_1} e^{At}BB^Te^{A^Tt}dt
\end{equation}
Вычислим грамиан управляемости для $t_1 = 3$ с помощью функции \texttt{gram}: 
\begin{equation}
    P(3) = \begin{bmatrix}
        18.12 & 10.97 & -11.64 \\ 
        10.97 & 7.48 & -8.48 \\ 
        -11.64 & -8.48 & 10.14 \\ 
    \end{bmatrix}
\end{equation}
Найдем собственные числа Грамиана управляемости: 
\begin{equation}
    \sigma(P(3)) = \{ 0.05, 1.94, 33.74 \}
\end{equation}
Все собственные числа Грамиана управляемости положительны, что говорит о том, что система является управляемой.


\subsection{Управление системой}
Найдем управление $u(t)$, которое будет переводить систему из состояния $x(0) = 0$ в состояние $x_1 = x(t_1) = \begin{bmatrix} -2 & -3 & 3 \end{bmatrix}^T$. 
\begin{equation}
    u(t) = B^Te^{A^T(t_1 - t)}P(t_1)^{-1}x_1
\end{equation}
Реализуем данное управление в MATLAB и проведем моделирование системы. 
\begin{figure}
    \centering
    \includegraphics[width=\textwidth]{media/plots/task1_control_signal.png}
    \caption{Управление системой}
    \label{fig:task1_control_signal}
\end{figure}
На рисунке \ref{fig:task1_control_signal} изображено управление системой.
\begin{figure}
    \centering
    \includegraphics[width=\textwidth]{media/plots/task1_states.png}
    \caption{Состояние системы}
    \label{fig:task1_state}
\end{figure}
На рисунке \ref{fig:task1_state} изображено состояние системы.

Видно, что система управляемая в соответствии с заданным управлением и переходит в заданное состояние. 
\FloatBarrier
\subsection{Вывод}
При исследовании системы, рассматриваемой в этом заднии, удалось показать, что она является 
полностью управляемой. Это было продемонстрировано с помощью критерия Калмана, через 
управляемость собственных значений и диагональную форму системы. Также был найден грамиан 
управляемости и проверены его собственные числа. Проведено моделирование системы с управлением, 
которое переводит систему в заданное состояние. Результаты моделирования показали, что система
управляема и управление работает корректно. 