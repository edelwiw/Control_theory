\section{Исследование управляемости по выходу}
Рассмотрим систему 
\begin{equation}
    \begin{cases}
        \dot{x} = Ax + Bu \\
        y = Cx + Du 
    \end{cases}
\end{equation}
где 
\begin{equation}
    \begin{array}{ccc}
        A = \begin{bmatrix}
            5 & -2 & 8 \\
            4 & -3 & 4 \\
            -4 & 0 & -7
        \end{bmatrix}, &
        B = \begin{bmatrix}
            -1 \\
            -3 \\
            3
        \end{bmatrix}, &
        C = \begin{bmatrix}
            0 & 3 & 5  \\ 
            0 & 14 & 9 \\
        \end{bmatrix} \\
    \end{array}
\end{equation}


\subsection{Диагональная форма системы}
Диагональную форму системы можно найти согласно следующим формулам:
\begin{equation}
    \begin{cases}
        \dot{\hat{x}} = P^{-1}AP\hat{x} + P^{-1}Bu \\ 
        y = CP\hat{x} + Du
    \end{cases} 
\end{equation}
где $P$ -- матрица собственных векторов матрицы $A$.
\begin{equation}
    P = \begin{bmatrix}
        -1 & -3+j & -3-j \\
        0 & -2 & -2 \\
        1 & 2 & 2
    \end{bmatrix}
\end{equation}

Тогда система примет вид: 
\begin{equation}
    \begin{cases}
        \dot{\hat{x}} = \begin{bmatrix}
            -3 & 0 & 0 \\
            0 & -1-2j & 0 \\
            0 & 0 & -1+2j
        \end{bmatrix} + \begin{bmatrix}
            0 \\
            0.75-1.75j \\
            0.75+1.75j \\
        \end{bmatrix} u \\ 
        y = \begin{bmatrix}
            5 & 4 & 4 \\
            9 & -10 & -10 \\
        \end{bmatrix} \hat{x} + Du 
    \end{cases}
    \label{eq:task4_diag}
\end{equation}
По системе в диагональной форме можно понять, что первое собственное число 
не является управляемым, так как первый компонент вектора $P^{-1}B$ равен нулю. 

При этом каждое из собственных чисел является наблюдаемым, так как все компоненты вектора $CP$ не равны нулю. 

\subsection{Управляемость по выходу}
Для того, чтобы определить, является ои система управляемой по выходу, необходимо посмотреть на 
матрицу управляемости по выходу $W_y$:
\begin{equation}
    \begin{array}{cc}
        U_y = \begin{bmatrix}
            CU & D
        \end{bmatrix} & 
        U = \begin{bmatrix}
            A, & AB, & A^2B
        \end{bmatrix} 
    \end{array}
\end{equation}
Для данной системы при $D = 0_{2\times 2}$ матрица управляемости по выходу равна: 
\begin{equation}
    \begin{array}{cc}
        U = \begin{bmatrix}
            -1  & 25  & -45 \\ 
            -3  & 17  & -19 \\ 
            3  & -17  & 19 \\ 
        \end{bmatrix} &
        U_y = \begin{bmatrix}
            6  & -34  & 38  & 0  & 0 \\ 
            -15  & 85  & -95  & 0  & 0 \\ 
        \end{bmatrix}
    \end{array}
\end{equation}
Определим ранг матрицы $U_y$:
\begin{equation}
    \text{rank}(U_y) = 1
\end{equation}
Так как размерность выхода равна 2, то система не является управляемой по выходу. 

Данная система не является полностью управляемой по выходу из-за того, что одно 
из собственных чисел не является управляемым, при этом входит в выход, согласно 
диагональной форме системы (см. систему \eqref{eq:task4_diag}).

Для того, чтобы сделать систему управляемой по выходу, необходимо изменить матрицу 
$D$ таким образом, чтобы выполнялся критерий управляемости по выходу, то есть, ранг 
матрицы $U_y$ должен быть равен 2. 

В данном случае подойдет любая ненулевая матрица $D$ размера $2\times 2$, например: 
\begin{equation}
    D = \begin{bmatrix}
        1 & 0 \\
        0 & 1
    \end{bmatrix}
\end{equation}
Тогда матрица управляемости по выходу будет равна:
\begin{equation}
    \hat{U}_y = \begin{bmatrix}
        6  & -34  & 38  & 1  & 0 \\ 
        -15  & 85  & -95  & 0  & 1 \\ 
        \end{bmatrix}
\end{equation}
и ее ранг будет равен 2.

\FloatBarrier
\subsection{Вывод}
В данном задании была рассмотрена \textit{полная} линейная система 
в форме В-С-В. Была найдена диагональная форма системы, исследована управляемость
по выходу и сделан вывод о том, что система не является управляемой по выходу. 
При этом, данную систему можно сделать полностью управляемой по выходу, добавив
ненулевую матрицу $D$ размера $2\times 2$. 