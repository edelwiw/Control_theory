\section{Управляемое подпространство}

Рассмотрим систему $\dot{x} = Ax + Bu$, где 
\begin{equation}
    \begin{array}{cc}
        A = \begin{bmatrix}
            5 & -2 & 8 \\
            4 & -3 & 4 \\
            -4 & 0 & -7
        \end{bmatrix}, &
        B = \begin{bmatrix}
            -1 \\
            -3 \\
            3
        \end{bmatrix}.
    \end{array}
\end{equation}

\subsection{Управляемость системы}
\subsubsection{Матрица управляемости}
Найдем матрицу управляемости $U = [B, AB, A^2B]$:
\begin{equation}
    U = \begin{bmatrix} 
        \begin{array}{c|c|c}
            \begin{bmatrix}
                -1 \\
                -3 \\
                3
            \end{bmatrix} & 
            \begin{bmatrix}
                5 & -2 & 8 \\
                4 & -3 & 4 \\
                -4 & 0 & -7
            \end{bmatrix} \times 
            \begin{bmatrix}
                -1 \\
                -3 \\
                3
            \end{bmatrix} &
            \begin{bmatrix}
                5 & -2 & 8 \\
                4 & -3 & 4 \\
                -4 & 0 & -7
            \end{bmatrix}^2 \times
            \begin{bmatrix}
                -1 \\
                -3 \\
                3
            \end{bmatrix}
        \end{array}   
    \end{bmatrix}
\end{equation}
\begin{equation}
    U = \begin{bmatrix}
    -1 & 25 & -45 \\ 
    -3 & 17 & -19 \\ 
    3 & -17 & 19 \\ 
    \end{bmatrix}
\end{equation}
Определим ранг матрицы управляемости:
\begin{equation}
    \text{rank}(U) = 2
\end{equation}
Так как ранг матрицы управляемости меньше размерности матрицы $A$, система не является полностью управляемой. 

\subsubsection{Управляемость собственных значений}
Определим управляемость собственных значений матрицы $A$. Для каждого собственного значения найдем матрицу Хаутуса $H_i = \begin{bmatrix} A - \lambda_i I & B \end{bmatrix}$ и определим ее ранг:
\begin{enumerate}
    \item $\lambda_1 = -3$: $H_1 = \begin{bmatrix}
        8 & -2 & 8 & -1\\
        4 & 0 & 4 & -3 \\
        -4 & 0 & -4 & 3
    \end{bmatrix}$, $\text{rank}(H_1) = 2$, собственное значение не управляемо.
    \item $\lambda_2 = -1-2j$: $H_2 = \begin{bmatrix}
        6+2j & -2 & 8 & -1\\
        4 & -2+2j & 4 & -3 \\
        -4 & 0 & -6+2j & 3
    \end{bmatrix}$, $\text{rank}(H_2) = 3$, собственное значение управляемо.
    \item $\lambda_3 = -1+2j$: $H_3 = \begin{bmatrix}
        6-2j & -2 & 8 & -1\\
        4 & -2-2j & 4 & -3 \\
        -4 & 0 & -6-2j & 3
    \end{bmatrix}$, $\text{rank}(H_3) = 3$, собственное значение управляемо.
\end{enumerate}

\subsubsection{Диагональная форма системы}
\begin{equation}
    \dot{\hat{x}} = \begin{bmatrix}
        -3 & 0 & 0 \\
        0 & -1-2j & 0 \\
        0 & 0 & -1+2j
    \end{bmatrix} \hat{x} + 
    \begin{bmatrix}
        0 \\
        \frac{3 - 7j}{4} \\ 
        \frac{3 + 7j}{4}
    \end{bmatrix} u
\end{equation}
Первое число в векторе $P^{-1}B$ равно нулю, значит, что первое состояние системы не является управляемым. 
Результаты совпали с результатами, полученными при анализе управляемости собственных значений через матрицу Хаутуса.

\subsection{Грамиан управляемости}
Найдем грамиан управляемости $P(t_1)$:
\begin{equation}
    P(t_1) = \int_{0}^{t_1} e^{At}BB^Te^{A^Tt}dt
\end{equation}
Вычислим грамиан управляемости для $t_1 = 3$ с помощью функции \texttt{gram}: 
\begin{equation}
    P(3) = \begin{bmatrix}
        26.65 & 13.37 & -13.37 \\ 
        13.37 & 8.28 & -8.28 \\ 
        -13.37 & -8.28 & 8.28 \\ 
    \end{bmatrix}
\end{equation}

Найдем собственные числа Грамиана управляемости:
\begin{equation}
   \sigma(P(3)) = \{0, 2.03, 41.17 \}
\end{equation}
Первое собственное число равно нулю, что говорит о том, что система не является полностью управляемой.

\subsection{Управляемое подпространство}
Выясним, принадлежат ли точки $x_1'$ и $x_1''$ управляемому подпространству:
\begin{equation}
    \begin{array}{cc}
        x_1' = \begin{bmatrix}
            -2 \\
            -3 \\
            3
        \end{bmatrix}, &
        x_1'' = \begin{bmatrix}
            -3 \\
            -3 \\
            4
        \end{bmatrix}
    \end{array}
\end{equation}

Для этого можно записать расширенную матрицу управляемости $U'$ и найти ранг этой матрицы:
\begin{equation}
    U' = \begin{bmatrix}
        -1 & 25 & -45 & -2 \\
        -3 & 17 & -19 & -3 \\
        3 & -17 & 19 & 3
    \end{bmatrix}
\end{equation}
\begin{equation}
    \text{rank}(U') = 2
\end{equation}

\begin{equation}
   U'' = \begin{bmatrix}
        -1 & 25 & -45 & -3 \\
        -3 & 17 & -19 & -3 \\
        3 & -17 & 19 & 4
    \end{bmatrix}
\end{equation}
\begin{equation}
    \text{rank}(U'') = 3
\end{equation}

Таким образом, можно сделать вывод, что точка $x_1'$ принадлежит управляемому подпространству, а точка $x_1''$ не принадлежит. В дальнейшем будем обозначать $x_1'$ как $x_1$.

\subsection{Управление системой}
Найдем управление $u(t)$, которое будет переводить систему из состояния $x(0) = 0$ в состояние $x_1 = x(t_1) = \begin{bmatrix} -2 & -3 & 3 \end{bmatrix}^T$. 
\begin{equation}
    u(t) = B^Te^{A^T(t_1 - t)}P(t_1)^{-1}x_1
\end{equation}
Реализуем данное управление в MATLAB и проведем моделирование системы.  
\begin{figure}
    \centering
    \includegraphics[width=\textwidth]{media/plots/task2_control_signal.png}
    \caption{Управление системой}
    \label{fig:task2_control_signal}
\end{figure}
На рисунке \ref{fig:task1_control_signal} изображено управление системой.
\begin{figure}
    \centering
    \includegraphics[width=\textwidth]{media/plots/task2_states.png}
    \caption{Состояние системы}
    \label{fig:task2_state}
\end{figure}
На рисунке \ref{fig:task1_state} изображено состояние системы.

Видно, что система управляемая в соответствии с заданным управлением и переходит в заданное состояние. 

\subsection{Вывод}
При исследовании системы, рассматриваемой в этом задании, получилось доказать, что она 
не является полностью управляемой. Это было продемонстрировано с помощью критерия Калмана,
через управляемость собственных значений и диагональную форму системы. При этом оказалось, 
что собственное число $\lambda_1 = -3$ не является управляемым. Также был найден грамиан
управляемости и проверены его собственные числа. Одно из собственных чисел равно нулю, что
говорит о том, что система не является полностью управляемой. 

Были рассмотрены две точки $x_1'$ и $x_1''$ и проверена их принадлежность управляемому
подпространству. Точка $x_1'$ принадлежит управляемому подпространству, а точка $x_1''$ не принадлежит. 
Проведено моделирование системы с управлением, которое переводит систему в заданное состояние.
Результаты моделирования показали, что система управляема и управление работает корректно.