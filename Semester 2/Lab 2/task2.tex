\section{Наблюдатель полного порядка}
Рассмотрим систему:
\begin{equation}
    \begin{array}{ll}
        \dot{x} = Ax\\
        y = Cx
    \end{array}
\end{equation}
где
\begin{equation}
    \begin{array}{ccc}
        A = \begin{bmatrix}
            -40 & 16 & 9 & 7 \\ 
            -64 & 25 & 14 & 12 \\ 
            -26 & 11 & 7 & 3 \\ 
            -48 & 18 & 14 & 8
        \end{bmatrix}, & 
        C = \begin{bmatrix}
            -3 \\ 2 \\ -2 \\ 1
        \end{bmatrix}^T
    \end{array}
\end{equation}
\subsection{Наблюдаемость собственных чисел}
Для определения наблюдаемости собственных чисел рассмотрим вещественную Жорданову форму системы:
\begin{equation}
    \begin{array}{ll}
        \dot{\hat{x}} = P^{-1}AP\hat{x}\\
        \hat{y} = C\hat{x}
    \end{array}
\end{equation}
Где $P$ -- матрица собственных векторов матрицы $A$, а $\hat{x} = P^{-1}x$.
\begin{equation}
    \begin{array}{ccc}
        \begin{bmatrix}
            -0.00  & -2.00  & 0.00  & 0.00 \\ 
            2.00  & 0.00  & 0.00  & 0.00 \\ 
            0.00  & 0.00  & -0.00  & -3.00 \\ 
            0.00  & 0.00  & 3.00  & 0.00 \\ 
        \end{bmatrix}, &
        P = \begin{bmatrix}
            1.14  & -0.05  & 1.13  & 0.14 \\ 
            1.74  & -0.22  & 1.84  & 0.14 \\ 
            0.87  & -0.11  & 0.71  & 0.00 \\ 
            1.41  & 0.00  & 1.41  & 0.00 \\ 
        \end{bmatrix}, & 
        C_j =\begin{bmatrix}
            -0.27 \\ -0.05  \\ 0.28  \\ -0.14 \\ 
        \end{bmatrix}^T
    \end{array}
\end{equation}
Таким образом, система является полностью наблюдаемой. 

\subsection{Наблюдатель полного порядка}
Рассмотрим наблюдатель полного порядка:
\begin{equation}
    \begin{array}{ll}
        \dot{\hat{x}} = A\hat{x} + L(C\hat{x} - y)\\
    \end{array}
\end{equation}

\subsubsection{Подбор спектра нааблюдателя}
Рассмотрим следующие варианты спектра наблюдателя:
\begin{enumerate}
    \item  $\sigma_1 = \{-1, -1, -1, -1\}$
    \item $\sigma_2 = \{-1, -10, -100, -100\}$
    \item $\sigma_3 = \{-1\pm2j, -1\pm3j\}$
\end{enumerate}
Для каждого из спектров найдем матрицу $L$ такую, чтобы спектр наблюдателя $\sigma(A - LC) = \sigma_i$. Для этого 
воспользуемся методом Аккермана. 


